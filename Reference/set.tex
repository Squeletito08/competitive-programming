
\section*{Set}

\begin{lstlisting}[language=C++]
	#include <bits/stdc++.h>
	using namespace std;
	
	int main() {
		set<int> s; // []
		
		// Insertar elementos
		
		s.insert(40); // [40]
		s.insert(10); // [10, 40]
		s.insert(20); // [10, 20, 40]
		s.insert(30); // [10, 20, 30, 40]
		s.insert(50); // [10, 20, 30, 40, 50]
		
		// Acceder e imprimir elementos
		for (const auto& elem : s) {
			cout << elem << " ";
		}
		cout << "\n";
		
		// Obtener el primer elemento
		cout << *s.begin() << "\n"; // 10
		
		// Obtener el ultimo elemento
		cout << *prev(s.end()) << "\n"; // 50
		
		// Usar count para verificar la existencia de un elemento
		cout << s.count(20) << "\n"; // 1
		cout << s.count(60) << "\n"; // 0
		
		// Eliminar un elemento
		s.erase(20); // [10, 30, 40, 50]
		
		// Verificar existencia de un elemento despues de eliminar
		if (s.find(20) == s.end()) {
			cout << "Elemento no encontrado.\n"; 
		}
		
		// Tamano del conjunto
		cout << s.size() << "\n"; // 4
		
		// Vaciar el conjunto
		s.clear();
		
		cout << "Conjunto vacio: " << (s.empty() ? "Si" : "No") << "\n"; // Si
		
		return 0;
	}
	
\end{lstlisting}
